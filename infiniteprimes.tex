\section{Computing $\mu_p(s)$ at infinite primes}\label{sec:infiniteprimes}

Infinite primes behave slightly differently than finite primes. In particular,
instead of weighting each Galois representation $G_\R = \Gal(\C/\R) \to D_n$ by
some prime number, we weight them simply by $1$. That is, if $S$ is a
collection of continuous homomorphisms $G_\R \to D_n$, then the weight
$\mu_\R^S$ is simply $|S|$.

Infinite primes also behave slightly differently than finite primes in the
allowance for ramification. In particular, we do {\em not} place restrictions
on the ramification of $K$ at infinity in our setting. That is, the image of
inertia (i.e., all of $G_\R$) is allowed to contain a rotation. Thus, the
weight $\mu_\R$ is simply $|D_n[2]|/|D_n|$, which is $(n+1)/2n$ when $n$ is odd
and $(n+2)/2n$ when $n$ is even.

On the other hand, we note that if we want to average over imaginary or real
quadratic fields $K$, then we note that if the image of inertia is a rotation
or trivial, then $K$ is a real quadratic field, else it is an imaginary
quadratic field. Thus, defining $\mu_\R^+$ (resp.~$\mu_\R^-$) to be the weight
at infinity for real (resp.~imaginary) quadratic fields, we have the following proposition.

\begin{prop}\label{prop:infiniteprimes}
  With $\mu_\R^\pm$ as above, we have $\mu_\R^+ = 1/2n$ when $n$ is odd and
$1/n$ when $n$ is even, and we have $\mu_\R^- = 1/2$ in either case.
\end{prop}
