\section{Introduction}\label{sec:introduction}

There has been quite a bit of work in the arithmetic statistics community over
the past decade on two conjectures: the Cohen-Lenstra Heuristics
\cite{CohenLenstra} and Bhargava's conjecture\footnote{This conjecture grew out of
the work of many people, notably starting with a conjecture of Linnick
\cite{Linnick} the that the number of $S_n$ fields with absolute discriminant
bounded by $X$ should be $\asymp X$. Bhargava's contribution was to interpret
the constant of proportionality, which is the most critical component for this
note.} on the density of discriminants of $S_n$ fields \cite{bhargavamass} (and
specifically Kedlaya's extensions \cite{kedlayamass}). The purpose of this note
is to show that these conjectures are {\em consistent}, and in fact, a particular
subset of Bhargava's conjecture for dihedral extensions would prove the Cohen-Lenstra
Heuristics.

Let $n > 1$ be an integer, $G \leq S_n$ be a transitive permutation group of
degree $n$, and $k$ a global field. Let $\sF = \sF(n, G, k)$ be the set of
degree $n$ field extensions $K$ of $k$ which have Galois closure $L$ with
$\Gal(L/k) \cong G$. Further, let $c: \sF \to \R_{\geq 0}$ be some {\em
counting function} and write \[ \sF_c(X) = \{ K \in \sF : c(K) < X \}. \]
Supposing that $\# \sF_c(X) < \infty$ for all $X$, we can ask for its
asymptotics.

\begin{question}
  Let $N_c(n, G, k; X) = N_c(X) = \# \sF_c(X)$. Does there exist some ``nice''
  function $f_c(X)$ (e.g., a polynomial in $X$ and $\log X$) such that \[ N_c(X) \sim f_c(X)? \]
\end{question}

This question has been answered in the affirmative for quite a few combinations
of $n$, $G$, $k$, and $c$. In particular, much work has been done when $c =
\Disc$ is the absolute value of the discriminant of the number field. In that
case, with $k = \Q$, the count was derived for $G$ abelian by M\"aki
\cite{Maki} and Wright \cite{WrightAbelian}, $n = 3$ and $G = S_3$ by
Davenport and Heilbronn \cite{DavenportHeilbronn}, and $n = 4, 5$ and $G = S_n$
by Bhargava \cite{BhargavaQuarticCount, BhargavaQuinticCount}. This was
recently extended by Bhargava, Shankar, and Wang \cite{BhargavaShankarWang} to
$k$ an arbitrary global field.

More general counting functions have also appeared in the literature.
Specifically, Wood extended M\"aki's and Wright's work \cite{melaniemass} to
greatly expand the type of counting function's by which abelian extensions of
arbitrary base fields can be counted.

Bhargava noted in \cite{bhargavamass} that all the known asymptotics tended to
appear as nice Euler products times some power of $X$ and some power of $\log
X$. Bhargava gave one description of these products in the case $G = S_n$, but
Kedlaya gave a more general interpretation for all $G$ when $c$ arises as the
conductor of a Galois representation \cite{kedlayamass}.  Specifically, fix a
finite group $G$ and a faithful representation $\eta: G \to \GL_m(\C)$. Then
$\eta$ defines a counting function $c = c_\eta$ on $\sF(n, G, k)$ which is the
global Artin conductor of $\eta$.

For each prime $\fp$ of the base field $k$, write $S_G(p)$ for the set of
Galois representations $\rho : \Gal(k^{\mathrm{sep}}/k) \to G$ and define the
{\em expected number} $E(N)$ of extensions with global Artin conductor $N =
\prod_p p^{e_p}$ to be \[ E(N) = \prod_p \frac{1}{|G|} \sum_{\substack{\rho \in
S_G(p) \\ \cond(\rho) = e_p}} 1 \] where $\cond(\rho)$ is the {\em local} Artin
conductor of $\rho$. The key assumption here is the independence of the various
primes. We then take the heuristic that
\[ \sum_{N \geq 1} E(N) \sim N_c(X). \]

We will be interested in studying dihedral extensions of $\Q$, which are
closely related to the class groups of quadratic fields (see
Section~\ref{sec:classgroups}).  Studying the characters attached to these
extensions (Section~\ref{sec:dihedralgroups}) and examining the Dirichlet
series attached to $E(N)$ (Section~\ref{sec:finiteprimes} for finite primes,
Section~\ref{sec:infiniteprimes} for the archimedean primes, and
Section~\ref{sec:dirichletseries} for putting them together), we prove the
following theorem in Section~\ref{sec:finalproof}.

\begin{thm}\label{thm:main}
  Bhargava's heuristics (as reinterpreted by Kedlaya) imply that
  \[ \frac{\sum_K \left| \Cl(K)^2[p] \right|}{\sum_K 1} \to 1 + p^{-1} \]
  for $K$ real and
  \[ \frac{\sum_K \left| \Cl(K)^2[p] \right|}{\sum_K 1} \to 2. \]
  for $K$ imaginary.
\end{thm}
