\section{The expected number of dihedral fields}\label{sec:dirichletseries}

At this point, the philosophy that Bhargava articulates \cite{bhargavamass}
would have us compute the ``expected number'' $E(N)$ of Galois $D_n$ extensions
$M/\Q$ with $\Cond(M) < X$ and whose inertia groups at all finite primes do not
contain a nontrivial rotation. Specifically, for $N = \prod_p p^{e_p}$, we set
\begin{equation}\label{eqn:endefn}
E(N) = \prod_p \frac{1}{|D_n|} \sum_{\substack{\rho \in S_{D_n}^{\refl}(p) \\ \cond(\rho) = e_p}} 1.
\end{equation}
The key assumption here is that the primes behave independently. Then we can set
\begin{equation}\label{eqn:phisdefn}
  \Phi(s) = \sum_{N \geq 1} \frac{E(N)}{N^s} = \prod_p \mu_p(s).
\end{equation}
As usual, the analytic properties of $\Phi(s)$ should tell us about the growth
of $\sum_{N \geq 1} E(N)$.

\subsection{$n$ odd}

\begin{thm}\label{thm:phiodd}
  When $n$ is odd, $\Phi(s)$ admits a meromorphic continuation to all $s \in
\C$ and has a simple pole at $s = 1$. The residue $\Res_{s=1}\Phi(s) =
\zeta(2)^{-1}$. Thus, by standard Tauberian theorems,
\[ \sum_{N \geq 1} E(N) \sim \zeta(2)^{-1} X. \]
\end{thm}
\begin{proof}

  From the computations of $\mu_p(s)$ in Section~\ref{sec:finiteprimes}, we see
that the $\Phi(s)$ defined in \eqref{eqn:phisdefn} converges absolutely for
$\Re(s) \geq 1$. Multiplying and dividing by $1 - p^{-s}$ at all $p$ we find
that for $\Re(s) \geq 1$
\[ \Phi(s) = \frac{1 - 2^{-s}}{1 - 2^{-s}} \cdot \mu_2(s) \cdot \prod_{p\text{ odd}} \frac{1 - p^{-2s}}{1 - p^{-s}}. \]
Then if we also multiply and divide by $1 + 2^{-s}$ we arrive at
\[ \Phi(s) = \frac{\mu_2(s)}{1 + 2^{-s}} \frac{\zeta(s)}{\zeta(2s)}. \]
This function, of course, has a meromorphic continuation to all $s \in \C$ with
its only zero or pole with $\Re(s) \geq 1$ being a simple pole at $s = 1$. From our computation of $\mu_2(1)$ in
Proposition~\ref{prop:finiteprimesevenp}, we find that
\[ \Res_{s = 1}\Phi(s) = \zeta(2)^{-1}, \]
and so we arrive at the theorem.
\end{proof}

\subsection{$n$ even}

In the case of $n$ even, the same computation as in the proof of
Theorem~\ref{thm:phieven} will yield a double pole coming from $\zeta(s)^2$ at
$s = 1$ and so we would expect a growth rate of $X \log X$. But recall from
????  that for our purposes we are less interested in the expected number of
$D_n$ fields with conductor $N$, but the expected number of $D_n$ fields
divided by the size of the two torsion in the class group of the quadratic
subfield.

However, genus theory teaches us that if $K$ has discriminant $N$, then $\left|
\Cl^+(K)[2] \right| = 2^{\omega(N)}$ where $\omega(N)$ is the number of prime
divisors of $N$ and $\Cl^+(K)$ is the {\em narrow class group} of $K$. Thus, we can
define
\begin{equation}\label{eqn:eprimendefn}
E'(N) = \frac{1}{\left| \Cl^+(K)[2] \right|} \prod_p \frac{1}{|D_n|} \sum_{\substack{\rho \in S_{D_n}^{\refl}(p) \\ \cond(\rho) = e_p}} 1.
\end{equation}
Forming the directly series
\[ \Phi'(s) = \sum_{N \geq 1} \frac{E'(N)}{N^s} = \prod_p \mu'_p(s). \]
Here we recall that for odd $p$, $\mu_p(s) = 1 + 2p^{-s}$ and so $\mu'_p(s) = 1 + p^{-s}$. For $p = 2$, we have that $\mu_2(s) = 1 + 2/2^{2s} + 4/2^{3s}$ and so $\mu'_2(s) = 1 + 1/2^{2s} + 2/2^{3s}$. Of course, these are {\em equal} to values of $\mu_p(s)$ when $n$ is odd, and so the same proof of Theorem~\ref{thm:phiodd} applies to the following theorem.

\begin{thm}\label{thm:phieven}
  When $n$ is even, $\Phi'(s)$ admits a meromorphic continuation to all $s \in
\C$ and has a simple pole at $s = 1$. The residue $\Res_{s=1}\Phi(s) =
\zeta(2)^{-1}$. Thus, by standard Tauberian theorems,
\[ \sum_{N \geq 1} E'(N) \sim \zeta(2)^{-1} X. \]
\end{thm}


\subsection{An aside on the narrow class group}

In \eqref{eqn:eprimendefn} we defined $E'(N)$ in terms of the {\em narrow}
class group. We would like to show that we could equally well define $E'(N)$ in
terms of the class group, and Theorem~\ref{thm:phieven} would remain unchanged.
More specifically, let
\begin{equation}\label{eqn:eprimendefn}
E''(N) = \frac{1}{\left| \Cl(K)[2] \right|} \prod_p \frac{1}{|D_n|} \sum_{\substack{\rho \in S_{D_n}^{\refl}(p) \\ \cond(\rho) = e_p}} 1.
\end{equation}

\begin{cor}\label{cor:classinsteadofnarrow}
  We have that
  \[ \sum_{N \geq 1} E''(N) \sim \frac{1}{2} \sum_{N \geq 1} E'(N). \]
\end{cor}

Recall the exact sequence
\begin{equation}\label{eqn:narrowclgrpexact}
1 \to F_\infty(K) \to \Cl^+(K) \to \Cl(K) \to 1,
\end{equation}

where $F_\infty(K) \leq C_2$.  This tells us that \[ \frac{1}{2} E'(N) \leq
E''(N) \leq E'(N). \]  But we also know that $F_\infty(K)$ is nontrivial if and
only if $K$ is real and the fundamental unit $\eps$ of $K$ has norm $1$.  But
if there is some prime $p \equiv 3 \pmod{4}$ with $p \mid \Disc(K)$, there can
be no such unit (or else $x^2 \equiv -1$ would have a solution modulo $p$), and
so $\Cl^+(K) \ne \Cl(K)$ for all such $K$. In fact, we have the following.

\begin{lem}\label{lem:splitcl2}
  Let $K$ be a real quadratic field and suppose $F_\infty = C_2$. Then
\eqref{eqn:narrowclgrpexact} splits if and only if there is a prime $p \equiv 3
\pmod{4}$ with $p \mid \Disc(K)$.
\end{lem}
\begin{proof}
  See, for instance, \cite[Th\'eor\`eme 8]{LouboutinThreeModFour}.
\end{proof}

So if all prime divisors $p \mid N$ satisfy $p \equiv 1 \pmod{4}$ then $E''(N)
= E'(N)$. On the other hand, if $N$ has a divisor which is congruent to $3
\pmod{4}$, then $E''(N) = \frac{1}{2} E'(N)$.

\begin{lem}
  Let $\sD(X)$ be the set of all positive fundamental discriminants less than
$X$ which have no prime divisor $\equiv 3 \pmod{4}$. Then
  \[ \sum_{N \in \sD(X)} E'(N) = o(X). \]
\end{lem}
\begin{proof}
  Forming the Diriclet series
  \begin{equation}\label{eqn:smalldirichtlet}
  \sum_{N \in \sD} \frac{E'(N)}{N^s} = \prod_{p \not\equiv 3 \pmod{4}} \mu'_p(s).
  \end{equation}
  As in the previous subsections, we may, upto a bounded constant, replace
$\mu'_2(s)$ with $1 + 2^{-s}$. Then, multiplying and dividing by $\prod_{p
\not\equiv 3 \pmod{4}} 1 - p^{-s}$ we find that \eqref{eqn:smalldirichtlet} is
equal to the product of a holomorphic function and the function \[ \prod_{p
\not\equiv 3 \pmod{4}} \left(1 - p^{-s}\right)^{-1}. \] However, this function
has been well-studied, e.g. in \cite[Satz 3]{RiegerNegPell}.  In particular,
this implies that the sum in question is $O(X/\sqrt{\log X})$.
\end{proof}

This lemma allows us to replace $E''(N)$ by $\frac{1}{2}E'(N)$ for all $N$, and
this completes the proof of Corollary~\ref{cor:classinsteadofnarrow}.
