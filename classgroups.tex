\section{Dihedral Fields and Class Groups}\label{sec:classgroups}

In this section we recall the relationship between dihedral extesions of $\Q$
and the class groups of quadratic fields. Recall that if $K = \Q(\sqrt{D})$ is a
quadratic extension of $\Q$ and $H(K)$ is the Hilbert class field of $K$ then
$H(K)/\Q$ is already Galois. So in particular, if $M/K$ is an unramified cyclic
extension of $K$ of degree $n$, then $M/\Q$ is Galois and $\Gal(M/\Q) \cong C_n
\rtimes C_2 \cong D_n$.  Then writing $C_n = \langle \sigma \rangle$ and $C_2 =
\langle \tau \rangle$, we see that $K$ is the fixed field of $\sigma$ and so,
since $M/K$ is unramifed, the inertia group $I_p$ at every prime $p$ of $\Q$
must have $I_p \cap \langle \sigma \rangle$ trivial.

However, $D_n$ contains the rotations $\sigma^i$ and the order $2$ reflections
$\sigma^i\tau$.  The product of any two reflections is a rotation, which is
nontrivial whenever the two multiplied reflections are distinct. This implies
that at all ramified primes $p$, $I_p$ must be an order $2$ group which
contains a reflection and the identity.

On the other hand, if $M/\Q$ is a Galois extension of degree $2n$ with Galois
group $\Gal(M/\Q) \cong D_n$, then $M$ is cyclic of degree $n$ over the fixed
field $K$ of $\sigma$.  This extension is unramified if and only if the inertia
groups $I_p \leq \Gal(M/\Q)$ at every prime $p$ have $I_p \cap \langle \sigma
\rangle = \{ e \}$. This proves the following theorem.

\begin{prop}\label{prop:nottoram}
  There is a one-to-one correspondence between on the one hand dihedral
extensions $M/\Q$ with $\Gal(M/\Q) \cong D_n$ with $I_p \cap \langle \sigma
\rangle = \{ e \}$ for all finite primes $p$ and on the other hand pairs $(K,
M)$ where $K$ is a quadratic extension of $\Q$ and $M/K$ is an unramified
cyclic extension.
\end{prop}

Next we would like to relate the number of unramified cyclic extensions $M/K$
of degree $n$ to the number of elements in $\Cl(K)[n]$. Recall that every
unramified extension of $K$ is a subextension of $H(K)$. Moreover,
$\Gal(H(K)/K) \cong \Cl(K)$. Thus, every unramified cyclic extension of degree
$n$ corresponds to a cyclic quotient of $\Cl(K)$ of order $n$. Since $\Cl(K)$
is a finite abelian group, duality assures us that cyclic quotients of order
$n$ are in one-to-one correspondence with cyclic subgroups of order $n$. But
the number of cyclic subgroups of $G$ of order $n$ is well-known. So we arrive
at the following.

\begin{prop}\label{prop:exttoclass}
  If $K$ is a quadratic field and $n = p_1^{e_1} \cdots p_r^{e_r}$, then the number
  of unramified cyclic extensions of degree $n$ of $K$ is equal to
  \begin{equation}\label{eqn:exttoclass}
    \prod_{i=1}^r \frac{\left|\Cl(K)[p_i^{e_i}]\right| - \left|\Cl(K)[p_i^{e_i-1}]\right|}{p^{e_r}(p - 1)}.
  \end{equation}
\end{prop}

For clarity, we note a few special cases when Propositions~\ref{prop:nottoram}
and \ref{prop:exttoclass} are combined. First, note that if $M/K/\Q$ is as in
Proposition~\ref{prop:nottoram}, then $\Disc(M/\Q) = \Disc(M/K)\Disc(K)^n =
\Disc(K)^n$ since $M/K$ is unramified. Let $\sqrt[n]{\Disc} : \sF(2n, D_n) \to
\R_{\geq 0}$ denote the function $M \mapsto \sqrt[n]{\Disc(M)}$. Then we write
$\sF^{\mathrm{refl}}(2n, D_n, \sqrt[n]{\Disc}; X)$ denote the set of dihedral
extensions $M/\Q$ with $\Gal(M/\Q) \cong D_n$ having $\sqrt[n]{\Disc(M)} < X$
and whose inertia groups $I_p$ at all finite primes contain no nontrivial
rotations. Then the two previous propositions immediately yield the following
corollary.

\begin{cor}\label{cor:oddprimeclass}
  If $n = p$ is prime, then
 \[ \# \sF^{\mathrm{refl}}(2n, D_n, \sqrt[n]{\Disc}; X) = \frac{1}{p-1} \sum_{K \in \sF(2, C_2, \Disc; X)} \left[\left|\Cl(K)[p]\right| - 1\right]. \]
\end{cor}
