\section{Computing $\mu_p(s)$ at finite primes}\label{sec:finiteprimes}

For any (finite) prime $p$, let $S_{D_n}^{\refl}$ the the set of continuous homomorphisms
$\rho : \Gal(\Qbar_p/\Qbar) \to D_n$ where the image of inertia under $\rho$ does not contain a rotation. Then define
\[ \mu_p(s) = \frac{1}{|D_n|} \sum_{\rho \in S_{D_n}^\refl} p^{-\cond(\rho)s} \]
where $\cond(\rho)$ is the conductor of $\rho$ when composed with faithful
two-dimensional representation $D_n \to \GL_2(\C)$. Note that this is
well-defined, since for any of these representations, the character will always
evaluate to $0$.

\begin{prop}\label{prop:finiteprimes}
  When $p$ is odd, the value of $\mu_p(s)$ is $1 + 2p^{-s}$ when $n$ is even,
and $1 + p^{-s}$ when $n$ is odd. Furthermore, $\mu_2(1) = 2 = 1 + 2/2$ when
$n$ is even, and $3/2 = 1 + 1/2$ when $n$ is odd.
\end{prop}

Note that the difference between the case $p = 2$ and $p \ne 2$ is that we do
not compute $\mu_2(s)$ for all $s$, but rather compute only $\mu_2(1)$. The
value of $\mu_2(s)$ is complicated, and we won't need it for the rest of this
note, so we simply avoid it.

We begin with the easiest case, when $(p, 2n) = 1$ and so $\rho$ is actually
{\em tame}.  Then $\rho$ factors through the tame Galois group
$\Gal(\Qbar_p/\Q_p)^\tame$ which is (topologically) generated by $g$ and $h$
with $hgh^{-1} = g^p$, and the inertia group is (topologically) generated by
$g$. Thus continuous homomorphisms $\rho : \Gal(\Qbar_p/\Q_p) \to D_n$ are in
one-to-one correspondence with pairs of elements $g, h \in D_n$ with $hgh^{-1}
= g^p$, and for any such pair, the value of $\cond(\rho)$ depends only on the
image of $g$.  On the other hand, $g$ must map to either $e$ or a reflection
under our assumptions. In either case, since $p$ is odd, $g^p = g$, and so we
are looking for elements $h \in Z_g(D_n)$ where $Z_g(D_n)$ is the centralizer
of $g$ in $D_n$. Then, abusing notation a bit and using the fact that $Z_e(D_n) = D_n$, we wish to compute
\[ \mu_p(s) = \frac{1}{|D_n|} \left[ |D_n| p^{-\cond(e) s} + \sum_{\sigma^i\tau} |Z_{\sigma^i\tau}(D_n)| p^{-\cond(g) s}\right]. \]

Now for any $g \in D_n$, we see that $\cond(g)$ is the number of eigenvalues of
$\rho(g)$ which are not equal to $1$. Thus, for $g = e$, we have $\cond(e) = 0$
for $g$ a reflection we have $\cond(g) = 1$. This reduces our sum to
\[ \mu_p(s) = 1 + \frac{p^{-s}}{|D_n|} \sum_{\sigma^i\tau} |Z_{\sigma^i\tau}(D_n)|. \]

When $n$ is odd, then all reflections are conjugate, and so the
orbit-stabilizer theorem tells us that $|Z_{\sigma^i\tau}(D_n)| = 2$ since half
of all the elements of $D_n$ are reflections. On the other hand, if $n$ is
even, then there are two conjugacy classes of reflections, so $|Z_{\sigma^i\tau}(D_n)| = 4$. Thus,
\[ \mu_p(s) = 1 + 2p^{-s} \]
when $n$ is even and
\[ \mu_p(s) = 1 + p^{-s} \]
when $n$ is odd.

Next we consider the case when $p$ is odd and $p \mid n$. Then it is no longer
guaranteed {\em a priori} that the higher ramification groups at $p$ are
trivial. However, we know that the quotients of the higher ramification groups
$I_i/I_{i+1}$ for $i \geq 1$ are products of cyclic groups of order $p$. But if
$I_1$ has a quotient which contains an element of order $p \ne 2$, then $I_1$
must contain a rotation. We are assuming that this is {\em not} the case in our
definition of $\mu_p$. This means that all the $\rho \in S_{D_n}^\refl$ are
{\em tame} and so we can repeat the above computation for all such $p$.

Finally, we come to the case when $p = 2$. First we consider the tame case.
Then again we're looking for pairs $g, h \in D_n$ with $hgh^{-1} = g^2$ and $g$
a reflection or the identity. But if $g$ is a reflection, then $g^2 = e \ne g$,
and so it must be that $g = e$. That is, if $\rho$ is at most tamely ramified
at $2$, then it is actually {\em unramified}. So the contribution from tamely
ramified extensions to $\mu_2(1)$ is simply $1$.

On the other hand, if $\rho$ is wildly ramified at $2$, then since $I_0$ is
already has order $2$, $I_1$ must be equal to $I_0$. Now we recall the
following fact about subgroups of dihedral groups.

Using Proposition~\ref{prop:dnsubgroups}, we see that the image of $\rho$ must
be either exactly $I_1$ or a dihedral group $D_m$ with $m \mid n$. In the case
where the image of $\rho$ is exactly $I_1$, we may simply look up the wildly
ramified, $C_2$ extensions of $\Q_2$ (e.g., with \cite{LMFDB}). Letting $t$
denote the number of nontrivial ramification groups (i.e., $I_t$ is the first
trivial ramification group of $\rho$), we see that there are two fields with $t
= 2$ and four fields with $t = 3$. In either case, the conductor is $t$ times
the number of nonzero eigenvalues of the image of a reflection under $\rho$,
which is $1$. Thus, the contribution from a single $\rho$ where the image is
exactly a reflection is $t$, and the total contribution to $\mu_2(1)$ is \[
\frac{1}{|D_n|} \sum_{\substack{\rho \in S_{D_n}^\refl \\ \image \rho\text{
reflection}}} 2^{-t(\rho)} = \frac{1}{|D_n|} \sum_{\text{reflections in $D_n$}}
\left[ 2 \cdot 2^{-2} + 4 \cdot 2^{-3} \right] = \frac{1}{2} \left[ \frac{1}{2}
+ \frac{1}{2} \right] = \frac{1}{2}. \]

What remains to compute, then, is the contribution to $\mu_2(1)$ coming from
$\rho$ whose image is a full dihedral group $D_m$. However, the image of wild
ramification is a {\em normal} nontrivial $2$-group in $D_m$. When $m$ is odd,
there are no such subgroups, and so if $n$ is odd, then our computation is
finished.

On the other hand, when $n$ is even, then {\em only} value of $m \mid n$ for
which the reflection subgroups are normal is $m = 2$.  Note that there are
$n/2$ subgroups of $D_n$ isomorphic to $D_2 \cong V_4$, namely, $\langle
\sigma^{n/2}, \sigma^i\tau \rangle$ with $0 \leq i < n/2$.

We can then look at the enumeration of all Galois quartic extensions of $\Q_2$
with Galois group $V_4$ and note that only three of them have an inertia group
isomorphic to $C_2$. Two of these have conductor $2^3$ and one has conductor
$2^2$.  Now $V_4$ has six automorphisms, and so for any of these three kernels,
there are six distinct maps $\Gal(\Qbar_2/\Q_2) \to V_4$ with that kernel. Of
these, four have the image of inertia as a reflection. So in total the
contribution to $\mu_2(1)$ is
\[ \frac{1}{|D_n|} \cdot \#\{ D_2 \leq D_n \} \cdot 4 \cdot \left[\frac{1}{4} + \frac{1}{8} + \frac{1}{8} \right]
  = \frac{1}{2n} \cdot \frac{n}{2} \cdot 4 \cdot \frac{1}{2}
  = \frac{1}{2}. \]

This completes the proof of Proposition~\ref{prop:finiteprimes}.
