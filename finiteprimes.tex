\section{Computing $\mu_p(s)$ at finite primes}\label{sec:finiteprimes}

For any (finite) prime $p$, let $S_{D_n}^{\refl}(p) = S_{D_n}^{\refl}$ the the set of continuous homomorphisms
$\rho : \Gal(\Qbar_p/\Qbar) \to D_n$ where the image of inertia under $\rho$ does not contain a rotation. Then define
\[ \mu_p(s) = \frac{1}{|D_n|} \sum_{\rho \in S_{D_n}^\refl} p^{-\cond(\rho)s} \]
where $\cond(\rho)$ is the conductor of $\rho$ when composed with faithful
two-dimensional representation $D_n \to \GL_2(\C)$. Note that this is
well-defined since the value of the character of every two-dimensional representation on reflections is $0$.

\begin{prop}\label{prop:finiteprimesoddp}
  When $p$ is odd and $n$ is even, $\mu_p(s) = 1 + 2p^{-s}$. When $p$ is odd
and $n$ is even, $\mu_p(s) = 1 + p^{-s}$.
\end{prop}
\begin{proof}
We begin with the easiest case, when $(p, 2n) = 1$ and so $\rho$ is actually
{\em tame}.  Then $\rho$ factors through the tame Galois group
$\Gal(\Qbar_p/\Q_p)^\tame$ which is (topologically) generated by $g$ and $h$
with $hgh^{-1} = g^p$, and the inertia group is (topologically) generated by
$g$. Thus continuous homomorphisms $\rho : \Gal(\Qbar_p/\Q_p) \to D_n$ are in
one-to-one correspondence with pairs of elements $g, h \in D_n$ with $hgh^{-1}
= g^p$, and for any such pair, the value of $\cond(\rho)$ depends only on the
image of $g$.  On the other hand, $g$ must map to either $e$ or a reflection
under our assumptions. In either case, since $p$ is odd, $g^p = g$, and so we
are looking for elements $h \in Z_g(D_n)$ where $Z_g(D_n)$ is the centralizer
of $g$ in $D_n$. Then, abusing notation a bit and using the fact that $Z_e(D_n) = D_n$, we wish to compute
\[ \mu_p(s) = \frac{1}{|D_n|} \left[ |D_n| p^{-\cond(e) s} + \sum_{\sigma^i\tau} |Z_{\sigma^i\tau}(D_n)| p^{-\cond(\sigma^i\tau) s}\right]. \]

Now for any $g \in D_n$, we see that $\cond(g)$ is the number of eigenvalues of
$\rho(g)$ which are not equal to $1$. Thus, for $g = e$, we have $\cond(e) = 0$
for $g$ a reflection we have $\cond(g) = 1$. This reduces our sum to
\[ \mu_p(s) = 1 + \frac{p^{-s}}{|D_n|} \sum_{\sigma^i\tau} |Z_{\sigma^i\tau}(D_n)|. \]

When $n$ is odd, all reflections are conjugate, and so the orbit-stabilizer
theorem tells us that $|Z_{\sigma^i\tau}(D_n)| = 2$ since half of all the
elements of $D_n$ are reflections. On the other hand, if $n$ is even, then
there are two conjugacy classes of reflections, so $|Z_{\sigma^i\tau}(D_n)| =
4$. And so the proposition is proved.

Next we consider the case when $p$ is odd and $p \mid n$. Then it is no longer
guaranteed {\em a priori} that the higher ramification groups at $p$ are
trivial. However, we know that the quotients of the higher ramification groups
$I_i/I_{i+1}$ for $i \geq 1$ are products of cyclic groups of order $p$. But if
$I_1$ has a quotient which contains an element of order $p \ne 2$, then $I_1$
must contain a rotation. We are assuming that this is {\em not} the case in our
definition of $\mu_p$. This means that all the $\rho \in S_{D_n}^\refl$ are
{\em tame} and so we can repeat the above computation for all such $p$.
\end{proof}

Finally, we come to the case when $p = 2$. Things are a bit more complicated,
but it turns out that the value of $\mu_2(1)$ has the same form as the value of
$\mu_p(1)$ for all odd $p$. Specifically, we have the following.

\begin{prop}\label{prop:finiteprimesevenp}
  When $n$ is odd we have
  \[ \mu_2(s) = 1 + \frac{1}{2^{-2s}} + \frac{2}{2^{-3s}} \]
  and when $n$ is even we have
  \[ \mu_2(s) = 1 + \frac{2}{2^{-2s}} + \frac{4}{2^{-3s}} \]
  In particular, $\mu_2(1) = 3/2 = 1 + 1/2$ when $n$ is odd and $\mu_2(1) = 2 = 1 + 2/2$ when $n$ is even.
\end{prop}
\begin{proof}

First we consider the tame case.  Then again we're looking for pairs $g, h \in
D_n$ with $hgh^{-1} = g^2$ and $g$ a reflection or the identity. But if $g$ is
a reflection, then $g^2 = e \ne g$, and so it must be that $g = e$. That is, if
$\rho$ is at most tamely ramified at $2$, then it is actually {\em unramified}.
So the contribution from tamely ramified extensions to $\mu_2(1)$ is simply
$1$.

In the wild case, since the inertia group $I_0$ has order $2$, the ramification
group $I_1 = I_0$. Then we can use Proposition~\ref{prop:dnsubgroups}, to see
that the image of $\rho$ must be either exactly $I_1$ or a dihedral group $D_m$
with $m \mid n$. In the case where the image of $\rho$ is exactly $I_1$, we may
simply look up the wildly ramified, $C_2$ extensions of $\Q_2$ (e.g., with
\cite{LMFDB}). Letting $t$ denote the number of nontrivial ramification groups
(i.e., $I_t$ is the first trivial ramification group of $\rho$), we see that
there are two fields with $t = 2$ and four fields with $t = 3$. In either case,
the conductor is equal to $t$.  So the total contribution to $\mu_2(s)$ is
\[ \frac{1}{|D_n|} \sum_{\substack{\rho \in S_{D_n}^\refl \\ \image \rho\text{ reflection}}} 2^{-t(\rho) s}
= \frac{1}{|D_n|} \sum_{\text{reflections in $D_n$}} \left[ 2 \cdot 2^{-2s} + 4 \cdot 2^{-3s}
\right]
= \frac{1}{2} \left[ \frac{2}{2^{-2s}} + \frac{4}{2^{-3s}} \right]
= \frac{1}{2^{-2s}} + \frac{2}{2^{-3s}}.
\]

What remains to compute is the contribution to $\mu_2(s)$ coming from
$\rho$ whose image is a full dihedral group $D_m$. However, the image of wild
ramification is a {\em normal} nontrivial $2$-group in $D_m$. When $m$ is odd,
there are no such subgroups, and so if $n$ is odd, then our computation is
finished.

On the other hand, when $n$ is even, then {\em only} value of $m \mid n$ for
which the reflection subgroups are normal is $m = 2$.  Note that there are
$n/2$ subgroups of $D_n$ isomorphic to $D_2 \cong V_4$, namely, $\langle
\sigma^{n/2}, \sigma^i\tau \rangle$ with $0 \leq i < n/2$.

We can then look at the enumeration of all Galois quartic extensions of $\Q_2$
with Galois group $V_4$ and note that only three of them have an inertia group
isomorphic to $C_2$. Two of these have conductor $3$ and one has conductor
$2$.  Now $V_4$ has six automorphisms, and so for any of these three kernels,
there are six distinct maps $\Gal(\Qbar_2/\Q_2) \to V_4$ with that kernel. Of
these, only four have the image of inertia as a reflection. So in total the
contribution to $\mu_2(s)$ is
\[ \frac{1}{|D_n|} \cdot \#\{ D_2 \leq D_n \} \cdot 4 \cdot
      \left[\frac{1}{2^{-2s}} + \frac{2}{2^{-3s}} \right]
  = \frac{1}{2n} \cdot \frac{n}{2} \cdot 4 \cdot \left[\frac{1}{2^{-2s}} + \frac{2}{2^{-3s}} \right]
  = \left[\frac{1}{2^{-2s}} + \frac{2}{2^{-3s}} \right]. \]

This completes the proof of the proposition.
\end{proof}
