\section{The character table of dihedral groups}\label{sec:dihedralgroups}

Recall from the introduction that one of the most interesting sources of
counting functions $c : \sF \to \R_{\geq 0}$ are those that arise as the Artin
conductor of $\Gal(L/\Q) \cong \Gamma \to \GL_m(\C)$ where $\Gamma \to
\GL_m(\C)$ is a fixed Galois representation. As we are interested in dihedral
extensions in this paper, we recall a couple basic facts about dihedral groups.

As in Section~\ref{sec:classgroups}, let $n \geq 2$ and let $D_n$ be the
dihedral group of symmetries of the regular $n$-gon generated by a rotation
$\sigma$ of order $n$ and a reflection $\tau$. In particular, note that $D_2
\cong V_4$. We first consider the subgroups of $D_n$.

\begin{prop}\label{prop:dnsubgroups}
  The subgroups of the dihedral group $D_n$ are the cyclic groups $C_m \leq
\langle \sigma \rangle$ with $m \mid n$, the dihedral groups $D_m$ with $m \mid
n$, and the subgroups $\langle \sigma^i \tau \rangle$ of order $2$ which
contain a single reflection. Of these, the subgroups of rotations are normal,
if $n = 2$, then all subgroups are normal, and if $n > 2$ is even, then the two
$D_{n/2}$ is normal.
\end{prop}

Next we recall that $\tau\sigma\tau = \sigma^{-1}$, and $\sigma\tau\sigma^{-1}
= \sigma^2\tau$ implies the following.

\begin{prop}\label{prop:dnconjclasses}
  When $n$ is even, the dihedral group $D_n$ has $n/2 + 3$ conjugacy classes
consisting of the pairs of rotations $\{ \sigma^i, \sigma^{-i} \}$ (with $i =
0$ and $i = n/2$ yielding singletons), and the two sets of reflections $\{
\sigma^{2i}\tau \mid 0 \leq i \leq n/2 \}$ and $\{ \sigma^{2i+1}\tau \mid 0
\leq i \leq n/2 \}$.

  When $n$ is odd, $D_n$ has $2 + (n - 1)/2$ conjugacy classes, consisting of
the pairs of rotations $\{ \sigma^i, \sigma^{-i} \}$ (with $i = 0$ yielding a
singleton), and the collection of all reflections $\{ \sigma^i\tau \mid 0 \leq
i \leq n - 1 \}$.
\end{prop}

Finally, we recall the character table of $D_n$.

\begin{prop}\label{prop:dnchartable}

  When $n$ is even, the dihedral group $D_n$ has the trivial representation,
three nontrivial one-dimensional representations (arising from the three $C_2$
quotients of $D_n$), and $n/2 - 1$ two-dimensional representations.

  When $n$ is odd, the dihedral group $D_n$ has the trivial representation, one
nontrivial one-dimensional representation, and $(n - 1)/2$ two-dimensional
representations.

  In both cases, in the two-dimensional represnetations, $\sigma^i$ maps to
either the identity or a nontrivial rotation of $\C^2$, and $\sigma^j\tau$ maps
to a reflection of $\C^2$.
\end{prop}
