\documentclass[11pt]{article}

\usepackage{amsfonts, amsmath, amssymb,latexsym, bbm, amsthm}

\usepackage[letterpaper, margin=1in]{geometry}

\usepackage{tikz}
\usetikzlibrary{positioning}
\usepackage{hyperref}

\newcommand{\Aut}{\operatorname{Aut}}
\newcommand{\Cl}{\operatorname{Cl}}
\newcommand{\Gal}{\operatorname{Gal}}
\newcommand{\Disc}{\operatorname{Disc}}
\newcommand{\GL}{\operatorname{GL}}
\newcommand{\Frac}{\operatorname{Frac}}
\newcommand{\Ind}{\operatorname{Ind}}
\newcommand{\Inn}{\operatorname{Inn}}
\newcommand{\Inv}{\operatorname{Inv}}
\newcommand{\Sel}{\operatorname{Sel}}
\newcommand{\SL}{\operatorname{SL}}
\newcommand{\Sym}{\operatorname{Sym}}
\newcommand{\SO}{\operatorname{SO}}
\newcommand{\Spec}{\operatorname{Spec}}
\newcommand{\Vol}{\operatorname{Vol}}
\newcommand{\Res}{\operatorname{Res}}
\newcommand{\N}{\operatorname{N}}
\newcommand{\Nm}{\operatorname{Nm}}
\newcommand{\rk}{\operatorname{rk}}
\newcommand{\ksep}{k^{\mathrm{sep}}}
\newcommand{\Qbar}{\Bar{\mathbb{Q}}}
\newcommand{\Cond}{\operatorname{Cond}}
\newcommand{\cond}{\operatorname{cond}}
\newcommand{\image}{\operatorname{im}}
\newcommand{\refl}{\mathrm{refl}}
\newcommand{\tame}{\mathrm{tame}}

\newcommand{\bOne}{\mathbbm{1}}
\newcommand{\C}{\mathbb{C}}
\newcommand{\CC}{{\rm C}}
\newcommand{\bF}{\mathbb{F}}
\newcommand{\Ss}{\mathcal{S}}
\newcommand{\bP}{\mathbb{P}}
\newcommand{\Q}{\mathbb{Q}}
\newcommand{\bQ}{\mathbb{Q}}
\newcommand{\R}{\mathbb{R}}
\newcommand{\Z}{\mathbb{Z}}
\newcommand{\bZ}{\mathbb{Z}}

\newcommand{\sF}{\mathcal{F}}
\newcommand{\W}{\mathcal{W}}
\newcommand{\M}{\mathcal{M}}
\newcommand{\sN}{\mathcal{N}}
\newcommand{\OO}{\mathcal{O}}
\newcommand{\cO}{\mathcal{O}}
\newcommand{\FF}{\mathcal{F}}
\newcommand{\UU}{\mathcal{U}}
\newcommand{\VV}{\mathcal{V}}
\newcommand{\F}{\mathbb{F}}

\newcommand{\fa}{\mathfrak{a}}
\newcommand{\fc}{\mathfrak{c}}
\newcommand{\ff}{\mathfrak{f}}
\newcommand{\fC}{\mathfrak{C}}
\newcommand{\fp}{\mathfrak{p}}
\newcommand{\fq}{\mathfrak{q}}

\newcommand{\eps}{\varepsilon}

\newcommand{\inj}{\hookrightarrow}
\newcommand{\jacobi}[2]{\left(\frac{#1}{#2}\right)}

\newcommand{\nosquare}{{\not\Box}}

\newtheorem{lem}{Lemma}[section]
\newtheorem{thm}[lem]{Theorem}
\newtheorem{cor}[lem]{Corollary}
\newtheorem{prop}[lem]{Proposition}
\newtheorem{conj}[lem]{Conjecture}
\newtheorem{lemma}{Lemma}

\theoremstyle{definition}
\newtheorem{defn}[lem]{Definition}
\newtheorem{rmk}[lem]{Remark}
\newtheorem{question}[lem]{Question}

\title{The Cohen-Lenstra Heuristics Follow from Bhargava's Mass Formula}
\author{Kevin H.~Wilson}

\begin{document}

\maketitle

\section{Introduction}\label{sec:introduction}

There has been quite a bit of work in the arithmetic statistics community over
the past decade on two conjectures: the Cohen-Lenstra Heuristics
\cite{CohenLenstra} and Bhargava's conjecture\footnote{This conjecture grew out of
the work of many people, notably starting with a conjecture of Linnick
\cite{Linnick} the that the number of $S_n$ fields with absolute discriminant
bounded by $X$ should be $\asymp X$. Bhargava's contribution was to interpret
the constant of proportionality, which is the most critical component for this
note.} on the density of discriminants of $S_n$ fields \cite{bhargavamass} (and
specifically Kedlaya's extensions \cite{kedlayamass}). The purpose of this note
is to show that these conjectures are {\em consistent}, and in fact, a particular
subset of Bhargava's conjecture for dihedral extensions would prove the Cohen-Lenstra
Heuristics.

Let $n > 1$ be an integer, $G \leq S_n$ be a transitive permutation group of
degree $n$, and $k$ a global field. Let $\sF = \sF(n, G, k)$ be the set of
degree $n$ field extensions $K$ of $k$ which have Galois closure $L$ with
$\Gal(L/k) \cong G$. Further, let $c: \sF \to \R_{\geq 0}$ be some {\em
counting function} and write \[ \sF_c(X) = \{ K \in \sF : c(K) < X \}. \]
Supposing that $\# \sF_c(X) < \infty$ for all $X$, we can ask for its
asymptotics.

\begin{question}
  Let $N_c(n, G, k; X) = N_c(X) = \# \sF_c(X)$. Does there exist some ``nice''
  function $f_c(X)$ (e.g., a polynomial in $X$ and $\log X$) such that \[ N_c(X) \sim f_c(X)? \]
\end{question}

This question has been answered in the affirmative for quite a few combinations
of $n$, $G$, $k$, and $c$. In particular, much work has been done when $c =
\Disc$ is the absolute value of the discriminant of the number field. In that
case, with $k = \Q$, the count was derived for $G$ abelian by M\"aki
\cite{Maki} and Wright \cite{WrightAbelian}, $n = 3$ and $G = S_3$ by
Davenport and Heilbronn \cite{DavenportHeilbronn}, and $n = 4, 5$ and $G = S_n$
by Bhargava \cite{BhargavaQuarticCount, BhargavaQuinticCount}. This was
recently extended by Bhargava, Shankar, and Wang \cite{BhargavaShankarWang} to
$k$ an arbitrary global field.

More general counting functions have also appeared in the literature.
Specifically, Wood extended M\"aki's and Wright's work \cite{melaniemass} to
greatly expand the type of counting function's by which abelian extensions of
arbitrary base fields can be counted.

Bhargava noted in \cite{bhargavamass} that all the known asymptotics tended to
appear as nice Euler products times some power of $X$ and some power of $\log
X$. Bhargava gave one description of these products in the case $G = S_n$, but
Kedlaya gave a more general interpretation for all $G$ when $c$ arises as the
conductor of a Galois representation \cite{kedlayamass}.  Specifically, fix a
finite group $G$ and a faithful representation $\eta: G \to \GL_m(\C)$. Then
$\eta$ defines a counting function $c = c_\eta$ on $\sF(n, G, k)$ which is the
global Artin conductor of $\eta$.

For each prime $\fp$ of the base field $k$, write $S_G(p)$ for the set of
Galois representations $\rho : \Gal(k^{\mathrm{sep}}/k) \to G$ and define the
{\em expected number} $E(N)$ of extensions with global Artin conductor $N =
\prod_p p^{e_p}$ to be \[ E(N) = \prod_p \frac{1}{|G|} \sum_{\substack{\rho \in
S_G(p) \\ \cond(\rho) = e_p}} 1 \] where $\cond(\rho)$ is the {\em local} Artin
conductor of $\rho$. The key assumption here is the independence of the various
primes. We then take the heuristic that
\[ \sum_{N \geq 1} E(N) \sim N_c(X). \]

We will be interested in studying dihedral extensions of $\Q$, which are
closely related to the class groups of quadratic fields (see
Section~\ref{sec:classgroups}).  Studying the characters attached to these
extensions (Section~\ref{sec:dihedralgroups}) and examining the Dirichlet
series attached to $E(N)$ (Section~\ref{sec:finiteprimes} for finite primes,
Section~\ref{sec:infiniteprimes} for the archimedean primes, and
Section~\ref{sec:dirichletseries} for putting them together), we prove the
following theorem in Section~\ref{sec:finalproof}.

\begin{thm}\label{thm:main}
  Bhargava's heuristics (as reinterpreted by Kedlaya) imply that
  \[ \frac{\sum_K \left| \Cl(K)^2[p] \right|}{\sum_K 1} \to 1 + p^{-1} \]
  for $K$ real and
  \[ \frac{\sum_K \left| \Cl(K)^2[p] \right|}{\sum_K 1} \to 2. \]
  for $K$ imaginary.
\end{thm}

\section{Dihedral Fields and Class Groups}\label{sec:classgroups}

In this section we recall the relationship between dihedral extesions of $\Q$
and the class groups of quadratic fields. Recall that if $K = \Q(\sqrt{D})$ is a
quadratic extension of $\Q$ and $H(K)$ is the Hilbert class field of $K$ then
$H(K)/\Q$ is already Galois. So in particular, if $M/K$ is an unramified cyclic
extension of $K$ of degree $n$, then $M/\Q$ is Galois and $\Gal(M/\Q) \cong C_n
\rtimes C_2 \cong D_n$.  Then writing $C_n = \langle \sigma \rangle$ and $C_2 =
\langle \tau \rangle$, we see that $K$ is the fixed field of $\sigma$ and so,
since $M/K$ is unramifed, the inertia group $I_p$ at every prime $p$ of $\Q$
must have $I_p \cap \langle \sigma \rangle$ trivial.

However, $D_n$ contains the rotations $\sigma^i$ and the order $2$ reflections
$\sigma^i\tau$.  The product of any two reflections is a rotation, which is
nontrivial whenever the two multiplied reflections are distinct. This implies
that at all ramified primes $p$, $I_p$ must be an order $2$ group which
contains a reflection and the identity.

On the other hand, if $M/\Q$ is a Galois extension of degree $2n$ with Galois
group $\Gal(M/\Q) \cong D_n$, then $M$ is cyclic of degree $n$ over the fixed
field $K$ of $\sigma$.  This extension is unramified if and only if the inertia
groups $I_p \leq \Gal(M/\Q)$ at every prime $p$ have $I_p \cap \langle \sigma
\rangle = \{ e \}$. This proves the following theorem.

\begin{prop}\label{prop:nottoram}
  There is a one-to-one correspondence between on the one hand dihedral
extensions $M/\Q$ with $\Gal(M/\Q) \cong D_n$ with $I_p \cap \langle \sigma
\rangle = \{ e \}$ for all finite primes $p$ and on the other hand pairs $(K,
M)$ where $K$ is a quadratic extension of $\Q$ and $M/K$ is an unramified
cyclic extension.
\end{prop}

Next we would like to relate the number of unramified cyclic extensions $M/K$
of degree $n$ to the number of elements in $\Cl(K)[n]$. Recall that every
unramified extension of $K$ is a subextension of $H(K)$. Moreover,
$\Gal(H(K)/K) \cong \Cl(K)$. Thus, every unramified cyclic extension of degree
$n$ corresponds to a cyclic quotient of $\Cl(K)$ of order $n$. Since $\Cl(K)$
is a finite abelian group, duality assures us that cyclic quotients of order
$n$ are in one-to-one correspondence with cyclic subgroups of order $n$. But
the number of cyclic subgroups of $G$ of order $n$ is well-known. So we arrive
at the following.

\begin{prop}\label{prop:exttoclass}
  If $K$ is a quadratic field and $n = p_1^{e_1} \cdots p_r^{e_r}$, then the number
  of unramified cyclic extensions of degree $n$ of $K$ is equal to
  \begin{equation}\label{eqn:exttoclass}
    \prod_{i=1}^r \frac{\left|\Cl(K)[p_i^{e_i}]\right| - \left|\Cl(K)[p_i^{e_i-1}]\right|}{p^{e_r}(p - 1)}.
  \end{equation}
\end{prop}

For clarity, we note a few special cases when Propositions~\ref{prop:nottoram}
and \ref{prop:exttoclass} are combined. First, note that if $M/K/\Q$ is as in
Proposition~\ref{prop:nottoram}, then $\Disc(M/\Q) = \Disc(M/K)\Disc(K)^n =
\Disc(K)^n$ since $M/K$ is unramified. Let $\sqrt[n]{\Disc} : \sF(2n, D_n) \to
\R_{\geq 0}$ denote the function $M \mapsto \sqrt[n]{\Disc(M)}$. Then we write
$\sF^{\mathrm{refl}}(2n, D_n, \sqrt[n]{\Disc}; X)$ denote the set of dihedral
extensions $M/\Q$ with $\Gal(M/\Q) \cong D_n$ having $\sqrt[n]{\Disc(M)} < X$
and whose inertia groups $I_p$ at all finite primes contain no nontrivial
rotations. Then the two previous propositions immediately yield the following
corollary.

\begin{cor}\label{cor:oddprimeclass}
  If $n = p$ is prime, then
 \[ \# \sF^{\mathrm{refl}}(2n, D_n, \sqrt[n]{\Disc}; X) = \frac{1}{p-1} \sum_{K \in \sF(2, C_2, \Disc; X)} \left[\left|\Cl(K)[p]\right| - 1\right]. \]
\end{cor}

\section{The character table of dihedral groups}\label{sec:dihedralgroups}

Recall from the introduction that one of the most interesting sources of
counting functions $c : \sF \to \R_{\geq 0}$ are those that arise as the Artin
conductor of $\Gal(L/\Q) \cong G \to \GL_m(\C)$ where $G \to \GL_m(\C)$ is a
fixed Galois representation. As we are interested in dihedral extensions in
this paper, we recall some basic facts about dihedral groups.

As in Section~\ref{sec:classgroups}, let $n \geq 2$ and let $D_n$ be the
dihedral group of symmetries of the regular $n$-gon generated by a rotation
$\sigma$ of order $n$ and a reflection $\tau$. We may choose $\sigma$ and
$\tau$ such that $\sigma\tau = \tau\sigma^{-1}$. Note that $D_2 \cong V_4$.

\begin{prop}\label{prop:dnsubgroups}
  The subgroups of the dihedral group $D_n$ are the cyclic groups $C_m \leq
\langle \sigma \rangle$ with $m \mid n$, the dihedral groups $D_m$ with $m \mid
n$, and the subgroups $\langle \sigma^i \tau \rangle$ of order $2$ which
contain a single reflection. Of these, the rotation subgroups are always
normal, and if $n = 2$, then all subgroups are normal. Further, if $n > 2$ is
even, then the two $D_{n/2}$ are normal as well.
\end{prop}

Next we recall that $\tau\sigma\tau = \sigma^{-1}$, and $\sigma\tau\sigma^{-1}
= \sigma^2\tau$ implies the following.

\begin{prop}\label{prop:dnconjclasses}
  When $n$ is even, the dihedral group $D_n$ has $n/2 + 3$ conjugacy classes
consisting of the pairs of rotations $\{ \sigma^i, \sigma^{-i} \}$ (with $i =
0$ and $i = n/2$ yielding singletons), and the two sets of reflections $\{
\sigma^{2i}\tau \mid 0 \leq i \leq n/2 \}$ and $\{ \sigma^{2i+1}\tau \mid 0
\leq i \leq n/2 \}$.

  When $n$ is odd, $D_n$ has $2 + (n - 1)/2$ conjugacy classes, consisting of
the pairs of rotations $\{ \sigma^i, \sigma^{-i} \}$ (with $i = 0$ yielding a
singleton), and the collection of all reflections $\{ \sigma^i\tau \mid 0 \leq
i \leq n - 1 \}$.
\end{prop}

Finally, we recall the character table of $D_n$.

\begin{prop}\label{prop:dnchartable}

  When $n$ is even, the dihedral group $D_n$ has the trivial representation,
three nontrivial one-dimensional representations (arising from the three $C_2$
quotients of $D_n$), and $n/2 - 1$ two-dimensional representations.

  When $n$ is odd, the dihedral group $D_n$ has the trivial representation, one
nontrivial one-dimensional representation, and $(n - 1)/2$ two-dimensional
representations.

  In both cases, in the two-dimensional representations, $\sigma^i$ maps to
either the identity or a nontrivial rotation of $\C^2$, and $\sigma^j\tau$ maps
to a reflection of $\C^2$.
\end{prop}

\section{Computing $\mu_p(s)$ at finite primes}\label{sec:finiteprimes}

For any (finite) prime $p$, let $S_{D_n}^{\refl}$ the the set of continuous homomorphisms
$\rho : \Gal(\Qbar_p/\Qbar) \to D_n$ where the image of inertia under $\rho$ does not contain a rotation. Then define
\[ \mu_p(s) = \frac{1}{|D_n|} \sum_{\rho \in S_{D_n}^\refl} p^{-\cond(\rho)s} \]
where $\cond(\rho)$ is the conductor of $\rho$ when composed with faithful
two-dimensional representation $D_n \to \GL_2(\C)$. Note that this is
well-defined, since for any of these representations, the character will always
evaluate to $0$.

\begin{prop}\label{prop:finiteprimes}
  When $p$ is odd, the value of $\mu_p(s)$ is $1 + 2p^{-s}$ when $n$ is even,
and $1 + p^{-s}$ when $n$ is odd. Furthermore, $\mu_2(1) = 2 = 1 + 2/2$ when
$n$ is even, and $3/2 = 1 + 1/2$ when $n$ is odd.
\end{prop}

Note that the difference between the case $p = 2$ and $p \ne 2$ is that we do
not compute $\mu_2(s)$ for all $s$, but rather compute only $\mu_2(1)$. The
value of $\mu_2(s)$ is complicated, and we won't need it for the rest of this
note, so we simply avoid it.

We begin with the easiest case, when $(p, 2n) = 1$ and so $\rho$ is actually
{\em tame}.  Then $\rho$ factors through the tame Galois group
$\Gal(\Qbar_p/\Q_p)^\tame$ which is (topologically) generated by $g$ and $h$
with $hgh^{-1} = g^p$, and the inertia group is (topologically) generated by
$g$. Thus continuous homomorphisms $\rho : \Gal(\Qbar_p/\Q_p) \to D_n$ are in
one-to-one correspondence with pairs of elements $g, h \in D_n$ with $hgh^{-1}
= g^p$, and for any such pair, the value of $\cond(\rho)$ depends only on the
image of $g$.  On the other hand, $g$ must map to either $e$ or a reflection
under our assumptions. In either case, since $p$ is odd, $g^p = g$, and so we
are looking for elements $h \in Z_g(D_n)$ where $Z_g(D_n)$ is the centralizer
of $g$ in $D_n$. Then, abusing notation a bit and using the fact that $Z_e(D_n) = D_n$, we wish to compute
\[ \mu_p(s) = \frac{1}{|D_n|} \left[ |D_n| p^{-\cond(e) s} + \sum_{\sigma^i\tau} |Z_{\sigma^i\tau}(D_n)| p^{-\cond(g) s}\right]. \]

Now for any $g \in D_n$, we see that $\cond(g)$ is the number of eigenvalues of
$\rho(g)$ which are not equal to $1$. Thus, for $g = e$, we have $\cond(e) = 0$
for $g$ a reflection we have $\cond(g) = 1$. This reduces our sum to
\[ \mu_p(s) = 1 + \frac{p^{-s}}{|D_n|} \sum_{\sigma^i\tau} |Z_{\sigma^i\tau}(D_n)|. \]

When $n$ is odd, then all reflections are conjugate, and so the
orbit-stabilizer theorem tells us that $|Z_{\sigma^i\tau}(D_n)| = 2$ since half
of all the elements of $D_n$ are reflections. On the other hand, if $n$ is
even, then there are two conjugacy classes of reflections, so $|Z_{\sigma^i\tau}(D_n)| = 4$. Thus,
\[ \mu_p(s) = 1 + 2p^{-s} \]
when $n$ is even and
\[ \mu_p(s) = 1 + p^{-s} \]
when $n$ is odd.

Next we consider the case when $p$ is odd and $p \mid n$. Then it is no longer
guaranteed {\em a priori} that the higher ramification groups at $p$ are
trivial. However, we know that the quotients of the higher ramification groups
$I_i/I_{i+1}$ for $i \geq 1$ are products of cyclic groups of order $p$. But if
$I_1$ has a quotient which contains an element of order $p \ne 2$, then $I_1$
must contain a rotation. We are assuming that this is {\em not} the case in our
definition of $\mu_p$. This means that all the $\rho \in S_{D_n}^\refl$ are
{\em tame} and so we can repeat the above computation for all such $p$.

Finally, we come to the case when $p = 2$. First we consider the tame case.
Then again we're looking for pairs $g, h \in D_n$ with $hgh^{-1} = g^2$ and $g$
a reflection or the identity. But if $g$ is a reflection, then $g^2 = e \ne g$,
and so it must be that $g = e$. That is, if $\rho$ is at most tamely ramified
at $2$, then it is actually {\em unramified}. So the contribution from tamely
ramified extensions to $\mu_2(1)$ is simply $1$.

On the other hand, if $\rho$ is wildly ramified at $2$, then since $I_0$ is
already has order $2$, $I_1$ must be equal to $I_0$. Now we recall the
following fact about subgroups of dihedral groups.

Using Proposition~\ref{prop:dnsubgroups}, we see that the image of $\rho$ must
be either exactly $I_1$ or a dihedral group $D_m$ with $m \mid n$. In the case
where the image of $\rho$ is exactly $I_1$, we may simply look up the wildly
ramified, $C_2$ extensions of $\Q_2$ (e.g., with \cite{LMFDB}). Letting $t$
denote the number of nontrivial ramification groups (i.e., $I_t$ is the first
trivial ramification group of $\rho$), we see that there are two fields with $t
= 2$ and four fields with $t = 3$. In either case, the conductor is $t$ times
the number of nonzero eigenvalues of the image of a reflection under $\rho$,
which is $1$. Thus, the contribution from a single $\rho$ where the image is
exactly a reflection is $t$, and the total contribution to $\mu_2(1)$ is \[
\frac{1}{|D_n|} \sum_{\substack{\rho \in S_{D_n}^\refl \\ \image \rho\text{
reflection}}} 2^{-t(\rho)} = \frac{1}{|D_n|} \sum_{\text{reflections in $D_n$}}
\left[ 2 \cdot 2^{-2} + 4 \cdot 2^{-3} \right] = \frac{1}{2} \left[ \frac{1}{2}
+ \frac{1}{2} \right] = \frac{1}{2}. \]

What remains to compute, then, is the contribution to $\mu_2(1)$ coming from
$\rho$ whose image is a full dihedral group $D_m$. However, the image of wild
ramification is a {\em normal} nontrivial $2$-group in $D_m$. When $m$ is odd,
there are no such subgroups, and so if $n$ is odd, then our computation is
finished.

On the other hand, when $n$ is even, then {\em only} value of $m \mid n$ for
which the reflection subgroups are normal is $m = 2$.  Note that there are
$n/2$ subgroups of $D_n$ isomorphic to $D_2 \cong V_4$, namely, $\langle
\sigma^{n/2}, \sigma^i\tau \rangle$ with $0 \leq i < n/2$.

We can then look at the enumeration of all Galois quartic extensions of $\Q_2$
with Galois group $V_4$ and note that only three of them have an inertia group
isomorphic to $C_2$. Two of these have conductor $2^3$ and one has conductor
$2^2$.  Now $V_4$ has six automorphisms, and so for any of these three kernels,
there are six distinct maps $\Gal(\Qbar_2/\Q_2) \to V_4$ with that kernel. Of
these, four have the image of inertia as a reflection. So in total the
contribution to $\mu_2(1)$ is
\[ \frac{1}{|D_n|} \cdot \#\{ D_2 \leq D_n \} \cdot 4 \cdot \left[\frac{1}{4} + \frac{1}{8} + \frac{1}{8} \right]
  = \frac{1}{2n} \cdot \frac{n}{2} \cdot 4 \cdot \frac{1}{2}
  = \frac{1}{2}. \]

This completes the proof of Proposition~\ref{prop:finiteprimes}.

\section{Computing $\mu_p(s)$ at infinite primes}\label{sec:infiniteprimes}

Infinite primes behave slightly differently than finite primes. In particular,
instead of weighting each Galois representation $G_\R = \Gal(\C/\R) \to D_n$ by
some prime number, we weight them simply by $1$. That is, if $S$ is a
collection of continuous homomorphisms $G_\R \to D_n$, then the weight
$\mu_\R^S$ is simply $|S|$.

Infinite primes also behave slightly differently than finite primes in the
allowance for ramification. In particular, we do {\em not} place restrictions
on the ramification of $K$ at infinity in our setting. That is, the image of
inertia (i.e., all of $G_\R$) is allowed to contain a rotation. Thus, the
weight $\mu_\R$ is simply $|D_n[2]|/|D_n|$, which is $(n+1)/2n$ when $n$ is odd
and $(n+2)/2n$ when $n$ is even.

On the other hand, we note that if we want to average over imaginary or real
quadratic fields $K$, then we note that if the image of inertia is a rotation
or trivial, then $K$ is a real quadratic field, else it is an imaginary
quadratic field. Thus, defining $\mu_\R^+$ (resp.~$\mu_\R^-$) to be the weight
at infinity for real (resp.~imaginary) quadratic fields, we have the following proposition.

\begin{prop}\label{prop:infiniteprimes}
  With $\mu_\R^\pm$ as above, we have $\mu_\R^+ = 1/2n$ when $n$ is odd and
$1/n$ when $n$ is even, and we have $\mu_\R^- = 1/2$ in either case.
\end{prop}


\section{Putting it all together}

At this point, Bhargava's philosophy in \cite{bhargavamass} and enumerated in
Section~\ref{sec:introduction} would have us compute the ``expected number''
$E(N)$ of Galois $D_n$ extensions $M/\Q$ with $\Cond(M) < X$ and whose inertia
groups at all finite primes do not contain a rotation. We begin by noting that
any such $N$ for which $E(N) \ne 0$ must be quite special.

\begin{lem}
  Let $M/K/\Q$ be a tower of extensions with $M/\Q$ Galois with $\Gal(M/\Q) \cong D_n$ and let $M/\Q$ be an unramified cyclic extension of degree $n$. Then $\Cond(M) = \Disc(K)$.
\end{lem}
\begin{proof}
  From our work in Section~\ref{sec:finiteprimes}, we know that at all odd
primes $p$, if $M/\Q$ is ramified at $p$, then the value of the Artin conductor
of $\rho_p$ is $1$. That is, $p \mid \Cond(M)$ and $p^2 \not\mid \Cond(M)$.  On
the other hand, since the image of inertia in $D_n$ is not a rotation, the
image of inertia in $\Gal(K/\Q) = \Gal(M/\Q)/\Gal(M/K)$ is nontrivial. Hence,
$p$ ramifies in $K$ and, of course, $p$ then exactly divides $\Disc(K)$.

If $p = 2$ and $p$ is ramified in $M$, then since there is only one slope in
the wild ramification of $M$, then there is at most one slope in the wild
ramification of $K$ at it is equal to the slope of wild ramification of $M$.
This completes the proof.
\end{proof}

Thus, $N$ must be a fundamental discriminant. Still following Bhargava, we set
\[ E(N) = \frac{1}{N} \]
when $n$ is odd and
\[ E(N) = \frac{2^{\omega(N)}}{N} \]
when $N$ is even. This is the product over each prime $p \mid N$ of $|D_n|^{-1}$ times $\sum_\rho p^{-\cond(\rho)}$ where the sum is over Galois representations $\rho : \Gal(\Qbar_p/\Q_p) \to D_n$ where the image of inertia is {\em exactly} a reflection (and not trivial). We can then write
\begin{equation}\label{eqn:phidefn}
\Phi(s) = \sum_{N \geq 1} \frac{E(N)}{N^s} = \prod_p \mu_p(s).
\end{equation}
Obviously this sum and this product ar absolutely convergent for $\Re(s) \gg 1$.

When $n$ is odd, we can multiply \eqref{eqn:phidefn} by $(1 - p^{-1})/(1 - p^{-1})$ and obtain
\[ \Phi(s) = \frac{\zeta(s)}{\zeta(2s)} \]
for $s \gg 1$. Thus, employing the usual Tauberian theorms, this implies that
\[ N_c(n, D_n, \Q; X) \sim \frac{1}{\zeta(2)}X \]
when $n$ is odd. On the other hand, when $n$ is even, we must multiply by $(1 - p^{-1})^2/(1 - p^{-1})^2$ and we obtain
\[ \Phi(s) = \zeta(s)^2 \prod_p \left( 1 - \frac{3}{p^{2s}} + \frac{2}{p^{3s}} \right). \]
Employing the same Tauberian theorems, we arrive at
\[ N_c(n, D_n, \Q; X) \sim \prod_p \left( 1 - \frac{3}{p^{2s}} + \frac{2}{p^{3s}} \right) X \log X. \]












\bibliography{final}{}
\bibliographystyle{plain}


\end{document}
